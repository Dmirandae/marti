\label{commands}
\index{commands}
\section{Text interface mode}
\label{tui_commands}
\index{commands:text interface}

\begin{center}
\begin{tabular}{lll}
	general & & \\
	\tui{l} & == list		&	List all tracks \\

	comparing *tracks* & & \\
		
	\tui{a} & == all	&	Track all the species using MSTs to compare\\			
	
	\tui{g} & == group	&	Track a group of contiguous species\\
	
	\tui{p} & == pair	&		Track a pair of species (contiguous or not)
\end{tabular}
\end{center}	  
In all the three cases a new MST=track will be created if the
species compared are congruent, while in a \tui{p}air comparison
only a single track could be created. In an \tui{a}ll/\tui{g}roup 
comparison, there could be more than one track.

	\begin{center}
\begin{tabular}{llp{6cm}}

	comparing *segments* & & \\
		
	\tui{s} & == all & Track all the species using segments to compare			\\
		
	\tui{t}  & == pair & Track a pair of species (contiguous or not)\\
	& & \\
	join & & \\
	\tui{u} & == join group	& Join a group of contiguous tracks if a given pair has a minimal congruence equal or higher than the
							minimal congruence inputted in the value file or
							the default value used in the TUI. If you want to join
							a group regardless of the minimal congruence value,
							reduce the min. cong. value to 0.0. After this late
							analysis, replace the original min. cong. value. \\
						
	\tui{j} & == join pair	&	Join two, contiguous or not, tracks independent of
						the minimal congruence value. \\
						

	congruence calculation & & \\
	
	\tui{o} & == congr. group &	Calculate the congruence by pairs of a contiguous group,
						as the output is a list, is easier to read a small 
						group than a large group. \\
							

	\tui{n}  & == congr. pair &	Calculate the congruence of a single pair. \\ 
	

	modify decision rules & & \\
	
	\tui{c} & == cut		&	Change the minimal cut value. \\

	\tui{r}  & == rules		&	Modify rules. \\

	\tui{m} & == min. cong.	&	Change the minimal congruence value.  \\
\end{tabular}
\end{center}

\section{Bash command mode}
\label{bash_commands}
\index{commands:bash}

A bash command is an instruction that will perform an analysis equal to the analysis made in the text user interface [see page \pageref{tui_commands}]. This mode is more appropriated to medium to huge data sets, when the analysis could last more than a few minutes or to test different parameter values. 

The commands are given in the equivalence of the TUI, with a short explanation.
\begin{center}
\begin{tabular}{lp{9cm}}
		\cmd{kmlgen} & == 	write to the kml file ONLY the generalized tracks (default). Must be in the file before the analysis. \\

		\cmd{kmlall} & == 	write to the kml file all, individual and generalized tracks. Must be in the file before the analysis. \\

		\cmd{croizat0} & == join (individual tracks), track  (individual tracks), and join (generalized tracks).\\
 & = \tui{u} (1 - \# individual tracks), \tui{a}, \tui{u} ( individual tracks + 1) total tracks\\

		\cmd{croizat1} & == 	track  (individual tracks) and 	join (generalized tracks)\\ 
& = \tui{a}, \tui{u} (individual tracks +1) total tracks \\
\end{tabular}
\end{center}

A sample command file is 

\begin{tabular}{l}
\cmd{set cv 1.5}\\
\cmd{set lmax 2.0}\\
\cmd{set lmin 0.6}\\
\cmd{set maxline 3.0}\\
\cmd{set ci 0.80}\\
\cmd{kmlall}\\
\cmd{croizat0}
\end{tabular}


So, these seven lines in a file will change the default parameters to 

\begin{tabular}{l}
\cmd{cv = 1.5}\\
\cmd{lmax = 2.0}\\
\cmd{lmin = 0.6}\\
\cmd{maxline = 3.0}\\
\cmd{ci = 0.80}
\end{tabular}

and will use these values to join the individual tracks if the congruence value of the two MSTs compared is larger than  0.80, then \MT  will track those resulting MST to obtain the generalized tracks and will eliminate (joint) those redundant tracks using the same rule used before (\cmd{ci = 0.80}), and will output the individual and the generalized tracks.   
