This manual is intended to be a clear and comprensive 
introduction to Panbiogeography and geometric track analyses, using the program \MT. 
The book is divided into four main chapters and some appendices.
First chapter is a general introduction to Panbiogeography and related terms, second chapter presents
\MT' algorithm while third chapter is devoted to commands used. The fourth and final chapter presents
some theorethical and empirical data and its analyses to explore \MT options.


\vspace{-7\baselineskip}\footnotetext{\url{http://ciencias.uis.edu.co/labsist/pantrack}}
\vspace{7\baselineskip}


The program was developed using FreePascal, as Pascal is an intuitive and easy to handdle language for programming, 
the debate is open to consider whether C/C++ or Phython/Perl/Java are more suitable languages for this kind of problem. As a free
software project, feel free to criticize the code or the algorithm, and if you improve it, we will be more than delighted and thankful. 


\vspace{-7\baselineskip}\footnotetext{\url{http://www.freepascal.org}}
\vspace{7\baselineskip}

This manual has been developed using \TL, maybe (in our humble opinion), the best environment to write a book or a handout.



We are grateful to many people, .....

We are open to new ideas or suggestions to improve our analyses, the program, and this manual itself. 

\vspace{-7\baselineskip}\footnotetext{\url{http://code.google.com/p/tufte-latex}}
\vspace{7\baselineskip}